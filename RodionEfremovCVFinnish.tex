\documentclass[11pt,a4paper,sans]{moderncv}

\moderncvstyle{classic}
\moderncvcolor{blue}
\renewcommand{\familydefault}{\sfdefault}
\nopagenumbers{}

\usepackage[finnish]{babel}
% Character encoding
\usepackage[utf8x]{inputenc}
\usepackage[scale=0.75]{geometry}

\firstname{Rodion}
\familyname{Efremov}
\title{Ansioluettelo}
\address{Pakkamestarinkatu 1G 113}{00520 Helsinki}
\mobile{+358~40~185~7884}
\email{rodion.efremov@helsinki.fi}
\homepage{http://www.cs.helsinki.fi/u/rodionef}

\begin{document}
\makecvtitle

\section{Koulutus}

\cventry{2008--2014}{Kandidaatin tutkinto, pääaine tietojenkäsittelytiede}{Helsigin yliopisto}{Helsinki}{\textit{keskiarvoinen arvosana 3.6/5.0}}{}
\cventry{2004--2008}{Ylioppilastutkinto}{Anjalankosken lukio/Kouvolan iltalukio}{Kouvola}{}{}

\section{Kandidaatintutkielma}

\cvitem{Otsikko}{\emph{Polunhakualgoritmit ja -järjestelmät}}
\cvitem{Lyhyt kuvaus}{Tutkin erilaisia polunhakualgoritmeja ja tapoja, joilla haku voidaan nopeuttaa. Näitä olivat mm. kaksisuuntaisuus, heuristinen haku, verkon esikäsittely ja rinnakkaisuus. Arvosana: 4/5.}
\cvitem{Osoite}{https://github.com/coderodde/bsc-thesis/blob/master/tutkielma.pdf?raw=true}

\section{Työkokemus}

\cventry{9.2009--4.2010}{Henkilökohtainen avustaja}{Vantaan kaupunki}{Vantaa}{}{
Kaupassa käynti, siivous ja ruuan laittaminen.}
\cventry{5.2009--9.2009}{Käyttöasentaja}{Oy VR-Rata Ab Kiskohitsaamo}{Kouvola}{}{Junaradan rakennus- ja kunnossapitotyöt.}
\cventry{6.2008--9.2008}{Kesätyölainen}{Oy VR-Rata Ab Kiskohitsaamo}{Kouvola}{}{Junaradan rakennus- ja kunnossapitotyöt.}
\cventry{4.2007--6.2007}{Talonmies}{Nuorten Kuntoutus Nurku Oy}{Kotka}{}{
Pihan ja talon hoitaminen, puutarha- ja puutyöt.}


\section{Kielitaito}

\cvitemwithcomment{Englanti}{erinomainen}{}
\cvitemwithcomment{Suomi}{erinomainen}{}
\cvitemwithcomment{Venäjä}{äidinkieli}{}

\section{Tekniset taidot}
\cvitem{Edistynyt}{Java,  C}
\cvitem{Hyvä}{Javascript, HTML5, CSS}
\cvitem{Keskitaso}{SQL}
\cvitem{Perustaso}{C++}

\section{Tekniset kiinnostuksen kohteet}
\cvitem{}{Yksikkö- ja integraatiotestaaminen,}
\cvitem{}{back-end -kehittäminen,}
\cvitem{}{verkkoalgoritmit,}
\cvitem{}{optimointialgoritmit,}
\cvitem{}{tietorakenteet,}
\cvitem{}{koneoppiminen,}
\cvitem{}{analytiikka.}

\section{Julkaisut}
\cvitem{\href{https://github.com/coderodde/lce-doc/blob/master/paper.pdf?raw=true}{Computing debt cuts leading to global zero-equity}}{Menetelmä, jolla voidaan laskea lainaverkossa sellaiset lyhennykset, että verkko kehittyy sellaiseen tilaan, jossa jokainen verkon solmu on yhtä paljon velkaa muille, kuin ollaan velkaa sille. Toteutin metodin \href{https://github.com/coderodde/lce}{\color{blue} Java-kielisenä kirjastona.}}

\section{Portfolio}

\cvitem{\href{https://github.com/coderodde/parallelsort}{\color{blue} Parallel radix sort}}{Huipputehokas, rinnakkainen lajittelualgoritmi, joka lajittelee mielivaltaiset objektit kokonaislukuavainten perusteella. Toteutettu Java-kielellä ja on tehokkaampti kuin monet muut lajittelualgoritmit.}

\cvitem{\href{https://github.com/coderodde/ACM-DL-Crawler}{\color{blue} ACM DL Crawler}}{Rinnakkainen crawleri, joka hakee sisällön verkosta ja organisoi sen tietokantaan.}

\cvitem{\href{http://www.cs.helsinki.fi/u/rodionef/pacman/}{\color{blue} Javascript PacMan}}{Javascript-kurssin puiteissa toteutettu klooni, joka toistaa alkuperäisen pelin grafiikat autenttisesti.}

\cvitem{\href{http://www.cs.helsinki.fi/u/rodionef/hide/Battleships/dist/Battleships.jar}{\color{blue} Java Battleships}}{Ohjelmoinnin harjoitustyökurssin tulos, joka osaa talentaa keskeneräisen pelin tilan ja palauttaa sen ensi käyttökerralla.}

\cvitem{\href{https://github.com/biblex2013/biblex}{\color{blue} biblex}}{Ohjelmistotuotantokurssilla toteutettu BiBLeX-viiteiden hallintaohjelmisto, joka tallentaa BibTeX-viitteet SQLite-tietokantaan, mikä mahdollistaa viiteiden lisäämisen, muokkaamisen, poistaamisen ja viemisen BiBTeX-formaatissa.}

\cvitem{\href{http://t-rodionef.users.cs.helsinki.fi/multilog/}{\color{blue} multilog}}{Tietokantasovelluskurssilla toteutettu keskustelufoorumi, joka tukee kolmen käyttäjäkategorian hierarkiaa, ilmoittaa käyttäjälle uusista viesteistä, tukee viestien merkintää (esim. tekstin lihavointi, jne.), osaa korjata epävalidin merkinnän ja mahdollistaa sisällön tekstiperustaista hakemista. Itse foorumi on jaettu aiheisiin, joista kukin sisältää useamman viestisäikeen.}

\section{Suosittelijat}
\cvitem{}{Ante Mulari, puh. 0505710781, email: ante.mulari@gmail.com, Vala Group Oy, Ohjelmistonkehitys ja -testaaminen.}

\end{document}