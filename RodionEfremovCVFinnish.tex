\documentclass[11pt,a4paper,sans]{moderncv}

\moderncvstyle{casual}
\moderncvcolor{blue}
\renewcommand{\familydefault}{\sfdefault}
\nopagenumbers{}

\usepackage[finnish]{babel}
% Character encoding
\usepackage[utf8x]{inputenc}
\usepackage[scale=0.75]{geometry}

\firstname{Rodion}
\familyname{Efremov}
\title{Ansioluettelo}
\address{Pakkamestarinkatu 1G 113}{00520 Helsinki}
\mobile{+358~40~185~7884}
\email{rodion.efremov@helsinki.fi}
\homepage{http://www.cs.helsinki.fi/u/rodionef}

\begin{document}
\makecvtitle

\section{Koulutus}

\cventry{2008--2014}{Kandidaatin tutkinto, pääaine tietojenkäsittelytiede}{Helsigin yliopisto}{Helsinki}{\textit{``grade point average'' 3.6/5.0}}{}
\cventry{2004--2008}{Ylioppilastutkinto}{Anjalankosken lukio/Kouvolan iltalukio}{Kouvola}{}{}

\section{Kandidaatintutkielma}

\cvitem{Otsikko}{\emph{Polunhakualgoritmit ja -järjestelmät}}
\cvitem{Lyhyt kuvaus}{Tutkielmassa tutkistelin erilaisia polunhakualgoritmeja ja tapoja, joilla haku voidaan nopeuttaa, kuten esimerkiksi kaksisuuntaisuus, heuristinen haku, verkon esikäsittely ja rinnakkaisuus. Arvosana: 4/5.}
\cvitem{Osoite}{https://github.com/coderodde/bsc-thesis/blob/master/tutkielma.pdf?raw=true}

\section{Kielitaito}

\cvitemwithcomment{Englanti}{sujuva}{}
\cvitemwithcomment{Suomi}{sujuva}{}
\cvitemwithcomment{Venäjä}{äidinkieli}{}

\section{Tekniset taidot}
\cvitem{Edistynyt}{\textsc{Java}, \LaTeX, \textsc{C}}
\cvitem{Hyvä}{\textsc{Javascript, HTML5, CSS}}
\cvitem{Perustaso}{\textsc{C++}}

\section{Tekniset kiinnostuksen kohteet}
\begin{itemize}
\item verkkoalgoritmit,
\item optimointialgoritmit,
\item tietorakenteet,
\item koneoppiminen.
\end{itemize}

\section{Julkaisut}
\cvitem{\href{https://github.com/coderodde/lce-doc/blob/master/paper.pdf?raw=true}{Computing debt cuts leading to global zero-equity}}{Menetelmä, jolla voidaan laskea lainaverkossa sellaiset lyhennykset, että verkko kehittyy sellaiseen tilaan, jossa jokainen verkon solmu on yhtä paljon velkaa muille, kuin ollaan velkaa sille.}

\section{Portfolio}

\cvitem{\href{http://www.cs.helsinki.fi/u/rodionef/pacman/}{Javascript PacMan}}{Javascript-kurssin puiteissa toteutettu klooni}
\cvitem{\href{http://www.cs.helsinki.fi/u/rodionef/hide/Battleships/dist/Battleships.jar}{Java Battleships}}{Ohjelmoinnin harjoitustyökurssin tulos}
\cvitem{\href{https://github.com/biblex2013/biblex}{biblex}}{Ohjelmistotuotantokurssilla toteutettu BiBLeX-viiteiden hallintaohjelmisto}
\cvitem{\href{http://t-rodionef.users.cs.helsinki.fi/multilog/}{multilog}}{Tietokantasovelluskurssilla toteutettu keskustelufoorumi}
\cvitem{\href{https://github.com/coderodde/parallelsort}{Parallel radix sort}}{Jo kahden suorittimen koneella lajittelee noin 5 kertaa nopeammin kuin \texttt{java.util.Arrays.sort}}
\cvitem{\href{https://github.com/coderodde/ACM-DL-Crawler}{ACM DL Crawler}}{Monisäikeinen crawleri, joka kerää tutkijoiden yhteistyöverkoston (engl. ``collaboration graph'')}
\cvitem{\href{https://github.com/coderodde/lce}{Loan cut equilibrium framework}}{Ainoaan julkaisuuni liittyvä pieni framework}
\cvitem{\href{http://codereview.stackexchange.com/users/58360/coderodde}{coderodde at Code Review}}{muuta alaan liittyvää toimintaa}

\end{document}